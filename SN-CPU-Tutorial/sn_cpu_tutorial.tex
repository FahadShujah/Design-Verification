\documentclass[a4paper,11pt]{article}

\setlength{\textwidth}{170mm}    \setlength{\textheight}{250mm}
\setlength{\parindent}{0Em}      \setlength{\parskip}{1ex}
\setlength{\topmargin}{0mm}      
\setlength{\headheight}{0mm}     \setlength{\headsep}{0mm} 
\setlength{\evensidemargin}{-10mm} \setlength{\oddsidemargin}{-10mm}

\newcommand{\stp}{\ensuremath{\mathit{STOP}}}
\newcommand{\ra}{\ensuremath{\rightarrow}}
\newcommand{\te}{\ensuremath{\sqsubseteq_T}}
\newcommand{\tr}[1]{\ensuremath{\mathit{traces(#1)}}}
\newcommand{\trc}[1]{\ensuremath{\mathit{\langle #1 \rangle}}}

\usepackage{times}

\begin{document}
\thispagestyle{empty}
\begin{center}
{\Large\bf DESIGN VERIFICATION Exercise 4: \\ 
\smallskip Introduction to SpecMan Elite}
\end{center}

\begin{quote}
  This exercise introduces you to SpecMan Elite. It gives you the opportunity
  to investigate some of the functionality of SpecMan Elite on a CPU example
  DUV. SpecMan will be used in the second DESIGN VERIFICATION assignment to
  write a sophisticated testbench for the calc1 design, so it is worth you
  investing some time into getting to know your tools better. SpecMan Elite is
  is part of the Incisive functional verification platform by Cadence.
  (\verb#http://www.cadence.com#)
  
  All files referred to in this exercise are available on the unit web
  pages. This sheet should be sufficient to get you started with SpecMan Elite. 
  If there are problems, please let me know. \\ Have fun! \\
  \hspace*{125mm} Kerstin
\end{quote}

\smallskip
\noindent {\Large\bf Getting Started}
\smallskip

\begin{enumerate}

\item Make sure you follow the EDA tools setup instructions on {\bf Getting
    started with SpecMan}.

\item To activate {\bf help} type:

\verb#cdnshelp &#

at a terminal window. This should open up the Cadence Help Browser. Go to the
``Edit'' menu, select the ``General'' tab and change the 
``Default Navigation View'' to ``Show Tree View on startup''.


\item Create a directory in your home directory for this exercise, 
and move into it:

\verb#mkdir SpecManTutorial#\\
\verb#cd SpecManTutorial#

Download the tutorial files from the unit web page, unzip and untar them.

\verb#gunzip sn_cpu_tutorial.tar.gz#\\
\verb#tar xfv sn_cpu_tutorial.tar#

Make sure you read the \verb#README#.

\end{enumerate}

\smallskip
\noindent {\Large\bf SpecMan Elite Tutorial}
\smallskip The file \verb#specman_tutorial.pdf# in the \verb#docs# directory
contains a comprehensive SpecMan Elite tutorial for an example CPU design. Both
the CPU and the testbench are modelled in e - this allows high-level modelling
and verification without the need for a simulator to run the tutorial. You are
expected to work your way through the tasks related to the tutorial during lab
sessions with an emphasis on learning how to use SpecMan, understanding e code
and the verification-specific features of the e language, as well as the
verification methodology.

\end{document}
