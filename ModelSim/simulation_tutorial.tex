\documentclass[a4paper,11pt]{article}

\setlength{\textwidth}{170mm}    \setlength{\textheight}{250mm}
\setlength{\parindent}{0Em}      \setlength{\parskip}{1ex}
\setlength{\topmargin}{0mm}      
\setlength{\headheight}{0mm}     \setlength{\headsep}{0mm} 
\setlength{\evensidemargin}{-10mm} \setlength{\oddsidemargin}{-10mm}

\newcommand{\stp}{\ensuremath{\mathit{STOP}}}
\newcommand{\ra}{\ensuremath{\rightarrow}}
\newcommand{\te}{\ensuremath{\sqsubseteq_T}}
\newcommand{\tr}[1]{\ensuremath{\mathit{traces(#1)}}}
\newcommand{\trc}[1]{\ensuremath{\mathit{\langle #1 \rangle}}}

\usepackage{times}

\begin{document}
\thispagestyle{empty}
\begin{center}
{\Large\bf DESIGN VERIFICATION Exercise 2: \\
\smallskip Introduction to the ModelSim Simulator}
\end{center}

\begin{quote}
  This exercise introduces you to the ModelSim simulator. It gives you the
  opportunity to investigate some of the functionality of ModelSim on two
  simple example designs. ModelSim is a simulator that can be used to verify
  the calculator design for the first DESIGN VERIFICATION assignment, so it is
  worth you investing some time into getting to know this simulator better. The
  exercises are very simple and are designed to familiarize you with the use of
  ModelSim and some of its features.
  
  All files referred to in this exercise are available on the unit web
  pages. This sheet should be sufficient to guide you through
  the exercise. If there are problems, please let me know. Have fun! \\
  \hspace*{125mm} Kerstin
\end{quote}

\smallskip
\noindent {\Large\bf Getting Started}
\smallskip

\begin{enumerate}
\item Make sure you have followed the {\bf EDA Software Access} instructions to set up the EDA tools for the unit. The tools run in the labs in MVB, 2.11 and 1.15, or via remote access.

\item Create a directory in your home directory for this exercise, 
and move into it, e.g.:

\verb#mkdir DV_Ex1#\\
\verb#cd DV_Ex1#

Now copy all (Verilog source) files mux421*.v from the unit web pages 
into the current directory. 

\item Now start ModelSim.

\verb#vsim &#

You should see two windows, one is a ``Welcome to ModelSim'' dialog and the
other one is entitled ``ModelSim SE ...''.  Complete the welcome dialog.
\end{enumerate}

\smallskip
\noindent {\Large\bf Familiarization with the Design(s) Under Verification and Testbench}
\smallskip

\begin{enumerate}
\item The files \verb#mux421_structural.v#, \verb#mux421_dataflow.v#,
  \verb#mux421_behavioural.v# \ contain the design of a 4-to-1 multiplexer
  coded in different styles. Familiarise yourself with the coding style used in
  each design.
  
\item The file \verb#mux421_gen.v# contains Verilog code that generates 6
  binary output signals which will be used as stimulus (input data) for our
  multiplexer designs.
  
\item The file \verb#mux421_check.v# contains Verilog code that checks signals.
  The task \verb#$monitor# continuously monitors the values of the variables or
  signals specified in the parameter list and displays all parameters in the
  list whenever the values of any one variable or signal changes.
  \verb#$monitor# only needs to be invoked once. Only one monitor list can be
  active at a time. If there is more than one \verb#$monitor# statement in your
  simulation, the last \verb#$monitor# statement will be the active statement,
  the earlier ones will be overridden.
  
  Monitoring is turned on by default at the beginning of the simulation and can
  be controlled during the simulation with the \verb#$monitoron# and
  \verb#$monitoroff# tasks.
  
  Usage \verb#$monitor# $(p1,p2,p3,...,pn)$\verb#;#
  
  The parameters $p1,p2,p3,...,pn$ can be variables, signal names, or quoted
  strings. NOTE: All Verilog system tasks appear in the form
  \verb#$<keyword>#.  You can find out more about Verilog system tasks such as
  \verb#$display# and \verb#$time# in the Verilog reference guide in your
  Evita\_Verilog tutorial.
  
\item The file \verb#mux421_example_testbench.v# contains {\em one\/} example
  testbench for the {\em structural} coding style description of the 4-to-1
  multiplexer design. Familiarise yourself with the testbench code.

\item In addition, there are source files that contain a faulty version of the
multiplexer, one for each coding style, and a file
\verb#mux421_faulty_testbench.v# which defines testbenches for each faulty
version. Familiarize yourself with these testbenches. Notice that the
StimulusGenerator and the Checker have been reused for each testbench - just
the DUV is different.

\end{enumerate}

\smallskip
\noindent {\Large\bf Simulating Verilog Designs}
\smallskip
\begin{enumerate}
\item To simulate a design with ModelSim, the Verilog files need to be compiled
  into a design library first. To do this, a design library has to be created
  (to hold the compilation results) and mapped to a physical library.
  
  In ModelSim, choose ``New:Library'' from the ``File'' menu. The ``Create a
  New Library'' window pops up. This window can be used to create a new library
  and a logical mapping to it, or to create a map to an existing library. The
  default library is \verb#work#.
  
  Select ``a new library and a logical mapping to it''.  If you don't want to
  use the default library, you can type a different library name, e.g.
  ``my\_work\_lib'', into the Library Name field, then select OK.  This creates
  a subdirectory named ``my\_work\_lib'' - your design library - within the
  current directory.
  
  Note that the Library Name field specifies the logical name of the library
  which is mapped to the physical name.  Observe that the new library is now
  shown in the ``Library'' pane on the left side of the ModelSim workspace area
  as an empty library.
  
  ModelSim can also be operated in a command-line mode from the Console window.
  Instead of using the ModelSim GUI to create the library and mapping, you
  could have just typed the following two commands:

  \verb#vlib my_work_lib#\\
  \verb#vmap my_work_lib my_work_lib#
  
  From now on, the command-line alternative will be given right at the
  end of an exercise step where possible.
  
\item After the working library has been created, the design files can be
  compiled.  Choose ``Compile'' from the ``Compile'' menu. The ``Compile HDL
  Source Files'' window appears showing the current directory, the default
  current working library (to be compiled into - by default) and available
  source files.  Make sure the working library is set to the one you've
  previously created. (The default is \verb#work# - other libraries are
available from the pull-down menu on the right.) The top-level file for
  compilation is the testbench file \verb#mux421_example_testbench.v# - select
  and compile it. Finish by clicking ``Done''.

  Command-line alternative: \verb#vlog -work my_work_lib mux421_example_testbench.v#
  
  Review the messages on the ModelSim Console window which show the compiled
  modules. Notice that \verb#mux421_structural_testbench# is the top-level
  module. Expand the working library (by double clicking on the + sign in front
  of it) in the Library pane on the left of the ModelSim workspace area - it
  now contains four compiled modules. (Enlarge the ``Name'' column to see their
  full names.)
  
\item Once the compilation of source files has been successfully completed, you
  need to load the design top-level unit for simulation. From the ``Simulate''
  menu choose ``Start Simulation. The ``Start Simulation'' window appears.
  Choose the ``Design'' tab which shows the currently available libraries.
  Expand your working library. You should see four units available for
  simulation. (You might need to enlarge the ``Name'' column again.) Select the
  \verb#mux421_structural_testbench# unit (the top-level unit). Do not yet press ``OK''.

\item To gain visibility of signals (which allows you to observe them in the
  Waveform viewer), you need to leave ``Enable optimization'' ticked, and set
  ``Apply full visibility to all modules (full debug mode)'' in the
  ``Visibility'' tab of the ``Optimization Options'' window. To get to this
  window, click the ``Optimization Options'' button next to the ``Enable
  optimization'' tick box at the bottom of the ``Start Simulation'' window.

\item Now press ``OK''. The ``Simulate'' window will close and information on
  the loading phase will be printed to the Console window. Review the messages
  printed to the Console.

  Command-line alternative: \\
  \verb#vsim my_work_lib.mux421_structural_testbench -voptargs=+acc#
  
  Notice also that the ModelSim workspace area has now changed. The ``sim''
  pane displays the design units loaded for simulation. It shows the
  hierarchical structure of the design and the instantiations. By default, only
  the top level of the hierarchy is expanded.
  
\item After the design has been loaded, you can see the signals of the design
  in the ``Objects'' window. (This assumes that you have applied the full
  visibility setting (full debug mode) as described above.)

  Command-line alternative: \verb#view signals#
    
  Select the root of the design in the ``sim'' pane.  The set of signals
  visible in the ``Objects'' window is automatically adjusted to the selection in
  the ``sim'' pane. 
  
  Compare the visible objects with the wire declaration in the module
  \verb#mux421_structural_testbench# \ contained in the file
  \verb#mux421_example_testbench.v# - all wires should be observable. Check
  this for the sub-units of the design.
  
\item Add all signals observable in the top-level unit
  \verb#mux421_structural_testbench# to the waveform viewer. In the ``Objects
  window use ``Select All'' from the ``Edit'' menu. Then right-click on the
  selected signals; choose ``Add to Wave'' and ``Selected Signals''. This will
  open the waveform viewer window.  You can also select specific signals for
  waveform viewing by dragging and dropping between the ``Objects'' and the
  ``Wave'' windows. 
  
\item After the desired signals have been added to the ``Wave'' browser window,
  the simulation can be run. To run all test vectors during one simulation step
  you should use ``Run'' and ``Run-All'' from the ``Simulate'' menu.

  Command-line alternative: \verb#run -all#
  
  Observe the messages printed to the Console. Also, observe the waveform
  changes - you might need to click the "Zoom Full" button from the ``Wave''
  window toolbar. In addition, a source code window will pop up indicating
  where the simulation stopped - you can ignore this for now.
    
\item To run the simulation again, first restart it by selecting ``Run'' and
  ``Restart'' from the ``Simulate'' menu. Observe that all signals are now set
  back in the ``Wave'' window. 
  
  Command-line alternative: \verb#restart -f#
  
\item Now run the simulation step-wise by repeatedly using ``Run'' and
  ``Run-Next'' from the ``Simulate'' menu.

  Command-line alternative: \verb#run -next#

\item Experiment with the simulator and waveform viewer in different settings
  and formats. Run the cursor over a waveform and observe the pop-up comments. 

\end{enumerate}

\smallskip
\noindent {\Large\bf Recording and comparing simulation results}
\smallskip

When the waveform viewer is active during simulation, the results of each
simulation run are automatically saved into a file called \verb#vsim.wlf# in
the current directory. (.wlf stands for wave log format.)  This file is
overwritten when a new simulation is run in the same directory. You can specify
a different name to save simulation results by selecting ``Datasets'' and
``sim'' from the ``File'' menu in the ``Wave'' window. The currently active
simulation is always prefixed by ``sim''. Note that a simulation session needs
to be finished with \verb#quit -sim# in order to produce a valid .wlf file.

\begin{enumerate}
\item Start ModelSim.
\item Create a new working library for this part of the exercise.
\item Compile \verb#mux421_example_testbench.v# - this should add 4 modules to
  your new working library. Check this in the Library pane.
\item Simulate \verb#mux421_structural_testbench#: Double-click on
  \verb#mux421_structural_testbench# in the Library pane to load this module
  for simulation. Right-click on \verb#mux421_structural_testbench# in the sim
  pane, then ``Add To Wave''. Now run the simulation with \verb#run -all#.
\item Save the simulation result: Select the dataset ``sim'' from  ``File''
  ``Datasets''. Select ``Save As''; call this dataset ``structural''. 

  Command line: \verb#dataset save sim structural.wlf#
\item Quit the simulation \verb#quit -sim#.
  
  The file \verb#structural.wlf# now contains the golden reference dataset for
  the structural version of our multiplexer design.

\item Now compile \verb#mux421_faulty_testbench.v# - this should add 6 more
  modules to your working library, giving a total of 10 modules in the
  library. Which modules are new?
\item Simulate \verb#mux421_faulty_structural_testbench#.
\item Save the simulation result of the faulty (structural) design. Call the
  dataset ``faulty\_structural''.
\item Quit this simulation.

\item Compare the simulation results by comparing the two waveforms: 

\begin{itemize} 
\item Use ``File - Datasets - Open'' to open the two \verb#.wlf# files just
  created. Two datasets appear, one called ``structural'' and one
  ``faulty\_structural''.
\item In the ``Wave'' tab, select ``Tools - Waveform Compare - Start
  Comparison''. Select ``structural'' as the Reference Dataset and specify
  ``faulty\_structural'' as the Test Dataset. Press ``OK''.
\item Now you need to specify the signals to be compared. Use ``Tools -
  Waveform Compare - Add - Compare by Region'' to compare all signals.
  Select \verb#/mux421_structural_testbench# as the Reference Region. Specify
  \verb#/mux421_faulty_structural_testbench# as the Test Region. Leave
  ``Compare Signals of Type'' unchanged - this will compare signals of all
  types. Press ``OK''. The Console should report ``Created 7 comparisons.'' The
  ``Wave'' window contains the comparisons created - investigate these.
\item Use ``Tools - Waveform Compare - Run Comparison'' to run the
  comparison. The Console should report ``Found 1 difference.'' The difference
  is marked in the ``Wave'' window - investigate.
\item You can use ``Tools - Waveform Compare - Differences - Show'' to get a
  report of the differences on the Console.
\item When you have experimented enough, use ``Tools - Waveform Compare -
  End Comparison'' to finish the comparison.
\end{itemize}
 
\end{enumerate}

\smallskip
\noindent {\Large\bf Experimenting with Testbenches}
\smallskip

Write your own testbenches to verify the behaviour of the dataflow and
behavioural versions of the 4-to-1 multiplexer. Initially, you should only have
to replace the DUV in the testbench file. Later you might want to add extra
code to the stimulus generator and the checker modules.

For each design, save the waveforms in suitably named files. Compare the
waveforms you've created - when comparing waveforms, specify the regions to be
compared and make sure corresponding signals have the same names. If you use
the same input stimulus, are the three (non-faulty) designs producing the same
output? The comparison results are printed to the Console window of ModelSim,
if differences are detected, the ``Wave'' window highlights the differences.


\noindent {\Large\bf Debugging faulty Multiplexers}
\smallskip

The three files \verb#mux421_*_faulty.v# contain faulty 4-to-1 multiplexer
designs.  For each design, build a testbench (or use
\verb#mux421_faulty_testbench.v#), simulate it, collect a waveform and compare
it to the waveform of the respective non-faulty design. You might want to
investigate the source code to see where the bug is - use \verb#diff# to
compare the source file to the corresponding non-faulty design source file if
the bug is not immediately obvious to you.

\smallskip
\noindent {\Large\bf A different Multiplexer design: Integer signals}
\smallskip

The file \verb#mux_int.v# contains the design of a multiplexer that selects
between two numeric inputs and also outputs a response indicating valid data.
Which coding style has been used: structural, dataflow or behavioural?

The file \verb#mux_int_test.v# contains a badly designed (i.e. not properly
modularised) testbench for the \verb#mux_int# module. Run the testbench, i.e.
simulate it. Observe the waveforms - try to understand the code in
\verb#mux_int_test.v#. Familiarise yourself with the \verb#always# statement
and the \verb#$display# task. In the ``Wave'' window you can change the
Radix of a signal, e.g. to decimal by right-clicking on the signal and
selecting the ``Radix'' option. Notice how the 32-bit wide signals are bundled
in the waveform - expand some to see the individual bits.

Redesign the testbench so that it contains one module that generates stimulus
and one module that acts as a checker. Experiment with different stimulus data,
run the simulation only for/until a set time, or introduce a bug into the
multiplexer source code and try to identify it from the waveform or the
testbench output.

\smallskip
\noindent {\Large\bf More on ModelSim}
\smallskip

The ``Help'' menu offers both PDF and HTML documentation on ModelSim under ``SE
Documentation''. You may find it useful to work through the {\bf ModelSim SE Tutorial}.
%,
%using the files provided in:
%\smallskip

%{\small
%\verb#/eda/mentor/2018-19/RHELx86/MODELSIM_-SE_2019.3/modeltech/examples/tutorials/verilog#
%}
\end{document}
